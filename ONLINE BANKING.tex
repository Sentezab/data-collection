\documentclass[]{report}
% Title Page
\title{ONLINE BANKING SYSTEM}
 
\begin{document}
	\maketitle
	
\chapter*{Introduction}
\section*{online banking system}
Internet is increasingly used by banks as a channel for receiving instructions and delivering their products and services to their customers. Different banks follow different levels for providing services on internet. Compared to banks abroad, online services still have a long way to go in terms of number of users and sufficient infrastructure in place. Various security options are in place or are being looked at; however, Certification Authority is still missing in India. Also there are various risks associated with internet banking such as Operational Risk, Security Risk, Cross Border Risks, Legal Risk, etc. The Basel Committee‟s Electronic Banking Group (EBG) in late 1999, tried to develop risk management guidance for Internet banking that will guide bankers and promote effective and consistent bank supervision around the world. The use of Information Technology in banking enables the banks to provide Any Time Banking, Customer Service, Telebanking, Home Banking, Plastic Card Services, etc., facilities. However, these facilities along with certain advantages these have certain disadvantages too. A successful internet banking solution offers:-  Exceptional rates on saving, cash deposits, free bill payment and rebates on ATM surcharges.  Credit cards with low rates  Easy online application for all accounts, including personal loans and mortgages  24 hours account access. 
\section*{Problem Statement}
The Scenario of Internet banking is changing the banking industry and is having the major effects on banking relationships. Internet banking involves use of Internet for delivery of banking products and services. It falls into four main categories, from Level 1 - minimum functionality sites that offer only access to deposit account data - to Level 4 sites - highly sophisticated offerings enabling integrated sales of additional products and access to other financial services- such as investment and insurance. In other words a successful Internet banking solution offers · Exceptional rates on Savings, CDs, and IRAs · Checking with no monthly fee, free bill payment and rebates on ATM surcharges · Credit cards with low rates · Easy online applications for all accounts, including personal loans and mortgages · 24 hour account access · Quality customer service with personal attention Internet banking, both as a medium of delivery of banking services and as a strategic tool for business development. At present, the total internet users in the country are estimated at 9 lakh

\end{document}          
